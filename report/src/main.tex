\documentclass[12pt]{report}
\usepackage{algorithm}
\usepackage{algpseudocode}
\usepackage{amsmath}
\usepackage{amssymb}
\usepackage{multicol}
\usepackage{tikz}


% change section numbering to remove leading chapter numbering
\makeatletter
\renewcommand \thesection {\@arabic\c@section.}
\renewcommand \thesubsection {\thesection\@arabic\c@subsection.}
\makeatother


% 5ab,14acd,18
\begin{document}

% title
\begin{titlepage}
    \begin{center}
        \vspace*{1cm}
        \Huge
        \textbf{Yet Another 2-opt} \\

        \vspace{0.5cm}
        \large
        Simple And Fast Approximation For \\
        The Traveling Salesperson Problem
            
        \vspace{1.5cm}

        \textbf{Jacob Thomsen} \\
        jakethom02@gmail.com \\
        \vspace{0.5cm}
        \textbf{John Tappen} \\
        jtappen@gmail.com \\
        \vspace{0.5cm}
        \textbf{Jonah Simmons} \\
        jonahksimmons@gmail.com \\

        \vfill
        Professor Tony Martinez \\
        Brigham Young University \\
        April 18, 2023
        \normalsize
    \end{center}
\end{titlepage}

\begin{multicols}{2}

    \section*{Abstract}
    The 2-opt algorithm is an optimization algorithm that can be applied to any approximate solution to the Traveling Salesperson Problem. We will explore and discuss our testing and variation on the algorithm as well as possible improvements to it.
    \section{Introduction}
    The Traveling Salesperson Problem (TSP) is arguably the most well know NP-Hard problem in the world. From its formalizing in the 19th century, it seems all possible solutions have been exhausted: from optimal exponential solutions to fast heuristic approximations. We will be focusing on an approximation that uses a greedy approach to find a fast solution. Then, we will improve upon it using an optimization known as \textit{2-opt}.

    \section{Greedy}
    Greedy algorithms solves problems with the "best now" mentality. It looks at all current options, assigns some quantifier to each, and chooses the best. For example, a greedy chess solver would protect its pieces or capture rather than sacrificing a piece for a checkmate in three moves.
    \subsection{Disadvantages}
    While greedy might seem very good, it has several flaws. As it does not take into account future decisions, it usually does not return the optimal for more complex problems, TSP being one of those. The solution from a greedy approach can be vastly worst than the optimal and, in theory, it almost random.
    \subsection{Advantages}
    Although it usually does not return an optimal, greedy approaches return good enough solutions for most problems. From the tests we ran, the greedy solutions were hard to beat by much. Greedy also usually returns a valid solution to most problems it is given.

    In addition to returning a good value, it does it very fast. The main issue with guaranteed optimal approaches is that it must take into account all other possibilities to ensure the best solution; this takes an unpractical amount of time.
\end{multicols}

\newpage
\subsection{Time Complexity}
\begin{algorithm}
\caption{Greedy algorithm}
\label{Greedy_Alg}
\begin{algorithmic}[1]
\Procedure{Greedy $\rightarrow$ path}{}
    \State let current = starting city
    \While{not visited all cities}
    \State let best = closest neighbor to \textit{current} that has not been visited
    \State path.add(best)
    \State current = best
    \EndWhile
\EndProcedure
\end{algorithmic}
\end{algorithm}

\begin{multicols}{2}
    Given the pseudo-code, we can see the time complexity is $O(V \times E)$, where $V$ is the number of cities and $E$ is the number of out edges. The outer loop goes through all nodes, $O(V)$, and each iteration looks at all neighbors, $O(E)$.
    \subsection{Purpose}
    The purpose of including the greedy algorithm is to offer a benchmark and comparison to the other algorithms. It produces fine results, but can be improved with a little bit of work. Due to its inefficiencies and speed, it will be used as the starting point for our 2-opt.

    \section{2-opt}
    Optimizations are very helpful in solving problems, especially very hard problems. One such optimization that works for TSP is \textit{2-opt} which makes minor changes to a given solution; in our case the initial solution is greedy. The basic idea of 2-opt is swapping two edges, if it would improve the overall cost, then re-ordering the rest of the graph to make it a valid path.

    This approach was chosen because of its flexibility and speed. It's flexible because it can be applied to any other approximate solution, no matter how it was solved. This "back end" optimization is especially helpful for very difficult problems, such as the TSP, and virtually has no drawbacks; whenever it is able to be used, it should b used.
    \subsection{Disadvantages}
    The issues with this approach are that it must have a valid solution to begin with and it only works for bi-directional graphs. For many real world problems, the bi-directional requirement can be a big problem and the initial solution that is chosen really impacts the output of 2-opt.
    \subsection{Advantages}
    Considering its shortcomings, 2-opt is still a very good approach. Since it is an optimization, it will always return a valid solution that is as good or better than your initial solution. In addition, 2-opt is rather fast.
\end{multicols}

\subsection{Time Complexity}
\begin{algorithm}
\caption{2-opt algorithm}
\label{2opt_Alg}
\begin{algorithmic}[1]
\Procedure{2opt $\rightarrow$ tour}{}
    \State let tour = initial solution
    \While{tour can be improved}
    \For{edge1, edge2 $\in$ tour}
    \If{swapping the destinations improves cost}
    \State swap them
    \State reverse intermediate edges
    \EndIf
    \EndFor
    \EndWhile
    \State return tour
\EndProcedure
\end{algorithmic}
\end{algorithm}

\begin{multicols}{2}
    As the pseudo-code shows, the number of iterations is unknown as the condition for the while loop is based on uncertain state. For each iteration, it must loop over all pairs of edges which is $O(E^2)$, where $E$ is the number of edges. So the time complexity of 2-opt is $O(P \times E^2)$, where $P$ is the unknown number of iterations.

    Since 2-opt has to use an initial solution, the total time complexity must account for that as well. For our implantation, we use the best of 10 runs of our greedy algorithm, so the time complexity would be $O(V \times E + P \times E^2)$, assuming $P$ is a constant, is equivalent to $O(V \times E + E^2)$.
\end{multicols}

%\bibliography{main} 
%\bibliographystyle{ieeetr}

\end{document}
